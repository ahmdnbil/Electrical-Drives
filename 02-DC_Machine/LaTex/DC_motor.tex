\documentclass[a4paper,11pt]{article}

%packages used
\usepackage{amsmath} 
\usepackage[utf8]{inputenc}
\usepackage{layout}

\begin{document}

\title{DC motor}
\maketitle

{\bf some notes}
\\
DC motor consists of Two coils, Armature and Field, Armature on Rotor and filed on stator they are perpendicular to each other
\\
\\
{\bf Armature} and {\bf Field} voltage general equations:- 
\[ V_a(t)=R_a i_a+L_a \frac{d i_a}{dt} + i_a \frac{d L_a}{dt} +M \frac{d i_f}{dt}+i_f \frac{dM}{dt} \]
\[ V_f(t)=R_f i_f+L_f \frac{d i_f}{dt} + i_f \frac{d L_f}{dt} +M \frac{d i_a}{dt}+i_a \frac{dM}{dt} \]
{\bf note}: \[ M=M_{af} \cos ( \theta )\] 
\\
For Armature voltage:

\begin{align*}
    M \frac{d i_f}{dt} = 0
\end{align*}

\begin{align*}
    i_a \frac{d L_a}{dt} = 0, \: \text{due to bruches} 
\end{align*}

\begin{align*}
i_f \frac{dM}{dt} = i_f \frac{dM}{d \theta} \frac{d \theta}{dt} = - M_{af} \sin(90) \omega_r i_f=\frac{p}{2} M_{af} \omega_m i_f 
\end{align*}
\\
\\
DC motor equations as following:
\begin{align}
    V_a(t)=R_a i_a + L_a \frac{d i_a}{dt} +M \omega i_f
\end{align}

\begin{align}
    V_f(t)=R_f i_f + L_f \frac{d i_f}{dt}
\end{align}

\begin{align}
    T_{em} = M i_f i_a
\end{align}

\begin{align}
    T_{em} = T_L + j \frac{d \omega_m}{dt} + \beta \omega_m
\end{align}


\end{document}